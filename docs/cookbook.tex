\documentclass[]{book}
\usepackage{lmodern}
\usepackage{amssymb,amsmath}
\usepackage{ifxetex,ifluatex}
\usepackage{fixltx2e} % provides \textsubscript
\ifnum 0\ifxetex 1\fi\ifluatex 1\fi=0 % if pdftex
  \usepackage[T1]{fontenc}
  \usepackage[utf8]{inputenc}
\else % if luatex or xelatex
  \ifxetex
    \usepackage{mathspec}
  \else
    \usepackage{fontspec}
  \fi
  \defaultfontfeatures{Ligatures=TeX,Scale=MatchLowercase}
\fi
% use upquote if available, for straight quotes in verbatim environments
\IfFileExists{upquote.sty}{\usepackage{upquote}}{}
% use microtype if available
\IfFileExists{microtype.sty}{%
\usepackage{microtype}
\UseMicrotypeSet[protrusion]{basicmath} % disable protrusion for tt fonts
}{}
\usepackage{hyperref}
\hypersetup{unicode=true,
            pdftitle={The Surgical Informatics Cookbook},
            pdfauthor={Surgical Informatics, University of Edinburgh},
            pdfborder={0 0 0},
            breaklinks=true}
\urlstyle{same}  % don't use monospace font for urls
\usepackage{natbib}
\bibliographystyle{apalike}
\usepackage{color}
\usepackage{fancyvrb}
\newcommand{\VerbBar}{|}
\newcommand{\VERB}{\Verb[commandchars=\\\{\}]}
\DefineVerbatimEnvironment{Highlighting}{Verbatim}{commandchars=\\\{\}}
% Add ',fontsize=\small' for more characters per line
\usepackage{framed}
\definecolor{shadecolor}{RGB}{248,248,248}
\newenvironment{Shaded}{\begin{snugshade}}{\end{snugshade}}
\newcommand{\AlertTok}[1]{\textcolor[rgb]{0.94,0.16,0.16}{#1}}
\newcommand{\AnnotationTok}[1]{\textcolor[rgb]{0.56,0.35,0.01}{\textbf{\textit{#1}}}}
\newcommand{\AttributeTok}[1]{\textcolor[rgb]{0.77,0.63,0.00}{#1}}
\newcommand{\BaseNTok}[1]{\textcolor[rgb]{0.00,0.00,0.81}{#1}}
\newcommand{\BuiltInTok}[1]{#1}
\newcommand{\CharTok}[1]{\textcolor[rgb]{0.31,0.60,0.02}{#1}}
\newcommand{\CommentTok}[1]{\textcolor[rgb]{0.56,0.35,0.01}{\textit{#1}}}
\newcommand{\CommentVarTok}[1]{\textcolor[rgb]{0.56,0.35,0.01}{\textbf{\textit{#1}}}}
\newcommand{\ConstantTok}[1]{\textcolor[rgb]{0.00,0.00,0.00}{#1}}
\newcommand{\ControlFlowTok}[1]{\textcolor[rgb]{0.13,0.29,0.53}{\textbf{#1}}}
\newcommand{\DataTypeTok}[1]{\textcolor[rgb]{0.13,0.29,0.53}{#1}}
\newcommand{\DecValTok}[1]{\textcolor[rgb]{0.00,0.00,0.81}{#1}}
\newcommand{\DocumentationTok}[1]{\textcolor[rgb]{0.56,0.35,0.01}{\textbf{\textit{#1}}}}
\newcommand{\ErrorTok}[1]{\textcolor[rgb]{0.64,0.00,0.00}{\textbf{#1}}}
\newcommand{\ExtensionTok}[1]{#1}
\newcommand{\FloatTok}[1]{\textcolor[rgb]{0.00,0.00,0.81}{#1}}
\newcommand{\FunctionTok}[1]{\textcolor[rgb]{0.00,0.00,0.00}{#1}}
\newcommand{\ImportTok}[1]{#1}
\newcommand{\InformationTok}[1]{\textcolor[rgb]{0.56,0.35,0.01}{\textbf{\textit{#1}}}}
\newcommand{\KeywordTok}[1]{\textcolor[rgb]{0.13,0.29,0.53}{\textbf{#1}}}
\newcommand{\NormalTok}[1]{#1}
\newcommand{\OperatorTok}[1]{\textcolor[rgb]{0.81,0.36,0.00}{\textbf{#1}}}
\newcommand{\OtherTok}[1]{\textcolor[rgb]{0.56,0.35,0.01}{#1}}
\newcommand{\PreprocessorTok}[1]{\textcolor[rgb]{0.56,0.35,0.01}{\textit{#1}}}
\newcommand{\RegionMarkerTok}[1]{#1}
\newcommand{\SpecialCharTok}[1]{\textcolor[rgb]{0.00,0.00,0.00}{#1}}
\newcommand{\SpecialStringTok}[1]{\textcolor[rgb]{0.31,0.60,0.02}{#1}}
\newcommand{\StringTok}[1]{\textcolor[rgb]{0.31,0.60,0.02}{#1}}
\newcommand{\VariableTok}[1]{\textcolor[rgb]{0.00,0.00,0.00}{#1}}
\newcommand{\VerbatimStringTok}[1]{\textcolor[rgb]{0.31,0.60,0.02}{#1}}
\newcommand{\WarningTok}[1]{\textcolor[rgb]{0.56,0.35,0.01}{\textbf{\textit{#1}}}}
\usepackage{longtable,booktabs}
\usepackage{graphicx,grffile}
\makeatletter
\def\maxwidth{\ifdim\Gin@nat@width>\linewidth\linewidth\else\Gin@nat@width\fi}
\def\maxheight{\ifdim\Gin@nat@height>\textheight\textheight\else\Gin@nat@height\fi}
\makeatother
% Scale images if necessary, so that they will not overflow the page
% margins by default, and it is still possible to overwrite the defaults
% using explicit options in \includegraphics[width, height, ...]{}
\setkeys{Gin}{width=\maxwidth,height=\maxheight,keepaspectratio}
\IfFileExists{parskip.sty}{%
\usepackage{parskip}
}{% else
\setlength{\parindent}{0pt}
\setlength{\parskip}{6pt plus 2pt minus 1pt}
}
\setlength{\emergencystretch}{3em}  % prevent overfull lines
\providecommand{\tightlist}{%
  \setlength{\itemsep}{0pt}\setlength{\parskip}{0pt}}
\setcounter{secnumdepth}{5}
% Redefines (sub)paragraphs to behave more like sections
\ifx\paragraph\undefined\else
\let\oldparagraph\paragraph
\renewcommand{\paragraph}[1]{\oldparagraph{#1}\mbox{}}
\fi
\ifx\subparagraph\undefined\else
\let\oldsubparagraph\subparagraph
\renewcommand{\subparagraph}[1]{\oldsubparagraph{#1}\mbox{}}
\fi

%%% Use protect on footnotes to avoid problems with footnotes in titles
\let\rmarkdownfootnote\footnote%
\def\footnote{\protect\rmarkdownfootnote}

%%% Change title format to be more compact
\usepackage{titling}

% Create subtitle command for use in maketitle
\providecommand{\subtitle}[1]{
  \posttitle{
    \begin{center}\large#1\end{center}
    }
}

\setlength{\droptitle}{-2em}

  \title{The Surgical Informatics Cookbook}
    \pretitle{\vspace{\droptitle}\centering\huge}
  \posttitle{\par}
    \author{Surgical Informatics, University of Edinburgh}
    \preauthor{\centering\large\emph}
  \postauthor{\par}
      \predate{\centering\large\emph}
  \postdate{\par}
    \date{2019-09-09}

\usepackage{booktabs}

\usepackage{amsthm}
\newtheorem{theorem}{Theorem}[chapter]
\newtheorem{lemma}{Lemma}[chapter]
\theoremstyle{definition}
\newtheorem{definition}{Definition}[chapter]
\newtheorem{corollary}{Corollary}[chapter]
\newtheorem{proposition}{Proposition}[chapter]
\theoremstyle{definition}
\newtheorem{example}{Example}[chapter]
\theoremstyle{definition}
\newtheorem{exercise}{Exercise}[chapter]
\theoremstyle{remark}
\newtheorem*{remark}{Remark}
\newtheorem*{solution}{Solution}
\begin{document}
\maketitle

{
\setcounter{tocdepth}{1}
\tableofcontents
}
\hypertarget{rules-of-posting}{%
\chapter{Rules of posting}\label{rules-of-posting}}

Rules of how to post here.

\hypertarget{intro}{%
\chapter{Introduction}\label{intro}}

You can label chapter and section titles using \texttt{\{\#label\}}
after them, e.g., we can reference Chapter \ref{intro}. If you do not
manually label them, there will be automatic labels anyway, e.g.,
Chapter \ref{methods}.

Figures and tables with captions will be placed in \texttt{figure} and
\texttt{table} environments, respectively.

\begin{Shaded}
\begin{Highlighting}[]
\KeywordTok{par}\NormalTok{(}\DataTypeTok{mar =} \KeywordTok{c}\NormalTok{(}\DecValTok{4}\NormalTok{, }\DecValTok{4}\NormalTok{, }\FloatTok{.1}\NormalTok{, }\FloatTok{.1}\NormalTok{))}
\KeywordTok{plot}\NormalTok{(pressure, }\DataTypeTok{type =} \StringTok{'b'}\NormalTok{, }\DataTypeTok{pch =} \DecValTok{19}\NormalTok{)}
\end{Highlighting}
\end{Shaded}

\begin{figure}

{\centering \includegraphics[width=0.8\linewidth]{cookbook_files/figure-latex/nice-fig-1} 

}

\caption{Here is a nice figure!}\label{fig:nice-fig}
\end{figure}

Reference a figure by its code chunk label with the \texttt{fig:}
prefix, e.g., see Figure \ref{fig:nice-fig}. Similarly, you can
reference tables generated from \texttt{knitr::kable()}, e.g., see Table
\ref{tab:nice-tab}.

\begin{Shaded}
\begin{Highlighting}[]
\NormalTok{knitr}\OperatorTok{::}\KeywordTok{kable}\NormalTok{(}
  \KeywordTok{head}\NormalTok{(iris, }\DecValTok{20}\NormalTok{), }\DataTypeTok{caption =} \StringTok{'Here is a nice table!'}\NormalTok{,}
  \DataTypeTok{booktabs =} \OtherTok{TRUE}
\NormalTok{)}
\end{Highlighting}
\end{Shaded}

\begin{table}[t]

\caption{\label{tab:nice-tab}Here is a nice table!}
\centering
\begin{tabular}{rrrrl}
\toprule
Sepal.Length & Sepal.Width & Petal.Length & Petal.Width & Species\\
\midrule
5.1 & 3.5 & 1.4 & 0.2 & setosa\\
4.9 & 3.0 & 1.4 & 0.2 & setosa\\
4.7 & 3.2 & 1.3 & 0.2 & setosa\\
4.6 & 3.1 & 1.5 & 0.2 & setosa\\
5.0 & 3.6 & 1.4 & 0.2 & setosa\\
\addlinespace
5.4 & 3.9 & 1.7 & 0.4 & setosa\\
4.6 & 3.4 & 1.4 & 0.3 & setosa\\
5.0 & 3.4 & 1.5 & 0.2 & setosa\\
4.4 & 2.9 & 1.4 & 0.2 & setosa\\
4.9 & 3.1 & 1.5 & 0.1 & setosa\\
\addlinespace
5.4 & 3.7 & 1.5 & 0.2 & setosa\\
4.8 & 3.4 & 1.6 & 0.2 & setosa\\
4.8 & 3.0 & 1.4 & 0.1 & setosa\\
4.3 & 3.0 & 1.1 & 0.1 & setosa\\
5.8 & 4.0 & 1.2 & 0.2 & setosa\\
\addlinespace
5.7 & 4.4 & 1.5 & 0.4 & setosa\\
5.4 & 3.9 & 1.3 & 0.4 & setosa\\
5.1 & 3.5 & 1.4 & 0.3 & setosa\\
5.7 & 3.8 & 1.7 & 0.3 & setosa\\
5.1 & 3.8 & 1.5 & 0.3 & setosa\\
\bottomrule
\end{tabular}
\end{table}

You can write citations, too. For example, we are using the
\textbf{bookdown} package \citep{R-bookdown} in this sample book, which
was built on top of R Markdown and \textbf{knitr} \citep{xie2015}.

\hypertarget{literature}{%
\chapter{Literature}\label{literature}}

Here is a review of existing methods.

\hypertarget{methods}{%
\chapter{Methods}\label{methods}}

We describe our methods in this chapter.

\hypertarget{machine-learning}{%
\chapter{Machine learning}\label{machine-learning}}

\hypertarget{deep-learning}{%
\section{Deep learning}\label{deep-learning}}

\hypertarget{pulling-images-from-redcap-directly-to-argodeep}{%
\subsection{Pulling images from REDCap directly to
argodeep}\label{pulling-images-from-redcap-directly-to-argodeep}}

\hypertarget{original-file-names}{%
\subsubsection{Original file names}\label{original-file-names}}

\begin{Shaded}
\begin{Highlighting}[]
\KeywordTok{library}\NormalTok{(REDCapR)}
\NormalTok{uri =}\StringTok{ "https://redcap.cir.ed.ac.uk/api/"}
\NormalTok{token =}\StringTok{ ""} \CommentTok{# API token here}
\NormalTok{record_list =}\StringTok{ }\DecValTok{1}\OperatorTok{:}\DecValTok{318}
\NormalTok{field_list =}\StringTok{ }\KeywordTok{c}\NormalTok{(}\StringTok{"photo"}\NormalTok{, }\StringTok{"photo_2"}\NormalTok{, }\StringTok{"photo_3"}\NormalTok{, }\StringTok{"photo_4"}\NormalTok{)}
\NormalTok{event_list =}\StringTok{ }\KeywordTok{c}\NormalTok{(}\StringTok{"wound_concerns_arm_2"}\NormalTok{, }\StringTok{"questionnaire_1_arm_2"}\NormalTok{,}
               \StringTok{"questionnaire_2_arm_2"}\NormalTok{, }\StringTok{"questionnaire_3_arm_2"}\NormalTok{)}
\NormalTok{directory =}\StringTok{ "wound_raw"} \CommentTok{# destination directory must exist already}


\ControlFlowTok{for}\NormalTok{(record }\ControlFlowTok{in}\NormalTok{ record_list)\{}
  \ControlFlowTok{for}\NormalTok{(field }\ControlFlowTok{in}\NormalTok{ field_list)\{}
    \ControlFlowTok{for}\NormalTok{(event }\ControlFlowTok{in}\NormalTok{ event_list)\{}
\NormalTok{      result =}\StringTok{ }
\StringTok{        }\KeywordTok{tryCatch}\NormalTok{(\{      }\CommentTok{# suppress breaking error when no image in slot}
          \KeywordTok{redcap_download_file_oneshot}\NormalTok{(}
            \DataTypeTok{record        =}\NormalTok{ record,}
            \DataTypeTok{field         =}\NormalTok{ field,}
            \DataTypeTok{redcap_uri    =}\NormalTok{ uri,}
            \DataTypeTok{token         =}\NormalTok{ token,}
            \DataTypeTok{event         =}\NormalTok{ event,}
            \DataTypeTok{overwrite     =} \OtherTok{TRUE}\NormalTok{,}
            \DataTypeTok{directory     =}\NormalTok{ directory}
\NormalTok{          )}
\NormalTok{        \}, }\DataTypeTok{error=}\ControlFlowTok{function}\NormalTok{(e)\{\})}
\NormalTok{    \}}
\NormalTok{  \}}
\NormalTok{\}}
\end{Highlighting}
\end{Shaded}

\hypertarget{named-from-redcap-record-id-and-event}{%
\subsubsection{Named from REDCap record ID and
event}\label{named-from-redcap-record-id-and-event}}

\begin{Shaded}
\begin{Highlighting}[]
\KeywordTok{library}\NormalTok{(REDCapR)}
\NormalTok{uri =}\StringTok{ "https://redcap.cir.ed.ac.uk/api/"}
\NormalTok{token =}\StringTok{ ""} \CommentTok{# API token here}
\NormalTok{record_list =}\StringTok{ }\DecValTok{1}\OperatorTok{:}\DecValTok{318}
\NormalTok{field_list =}\StringTok{ }\KeywordTok{c}\NormalTok{(}\StringTok{"photo"}\NormalTok{, }\StringTok{"photo_2"}\NormalTok{, }\StringTok{"photo_3"}\NormalTok{, }\StringTok{"photo_4"}\NormalTok{)}
\NormalTok{event_list =}\StringTok{ }\KeywordTok{c}\NormalTok{(}\StringTok{"wound_concerns_arm_2"}\NormalTok{, }\StringTok{"questionnaire_1_arm_2"}\NormalTok{,}
               \StringTok{"questionnaire_2_arm_2"}\NormalTok{, }\StringTok{"questionnaire_3_arm_2"}\NormalTok{)}
\NormalTok{directory =}\StringTok{ "wound_named"} \CommentTok{# destination directory must exist already}

\ControlFlowTok{for}\NormalTok{(record }\ControlFlowTok{in}\NormalTok{ record_list)\{}
  \ControlFlowTok{for}\NormalTok{(field }\ControlFlowTok{in}\NormalTok{ field_list)\{}
    \ControlFlowTok{for}\NormalTok{(event }\ControlFlowTok{in}\NormalTok{ event_list)\{}
\NormalTok{      file_name =}\StringTok{ }\KeywordTok{paste0}\NormalTok{(record, }\StringTok{"_"}\NormalTok{, field, }\StringTok{"_"}\NormalTok{, event, }\StringTok{".jpg"}\NormalTok{)}
\NormalTok{      result =}\StringTok{ }
\StringTok{        }\KeywordTok{tryCatch}\NormalTok{(\{}
          \KeywordTok{redcap_download_file_oneshot}\NormalTok{(}
            \DataTypeTok{record        =}\NormalTok{ record,}
            \DataTypeTok{field         =}\NormalTok{ field,}
            \DataTypeTok{redcap_uri    =}\NormalTok{ uri,}
            \DataTypeTok{token         =}\NormalTok{ token,}
            \DataTypeTok{event         =}\NormalTok{ event,}
            \DataTypeTok{overwrite     =} \OtherTok{TRUE}\NormalTok{,}
            \DataTypeTok{directory     =}\NormalTok{ directory,}
            \DataTypeTok{file_name     =}\NormalTok{ file_name}
\NormalTok{          )}
\NormalTok{        \}, }\DataTypeTok{error=}\ControlFlowTok{function}\NormalTok{(e)\{\})}
\NormalTok{    \}}
\NormalTok{  \}}
\NormalTok{\}}
\end{Highlighting}
\end{Shaded}

\hypertarget{final-words}{%
\chapter{Final Words}\label{final-words}}

We have finished a nice book.

\hypertarget{plotting}{%
\chapter{Plotting}\label{plotting}}

\hypertarget{gghighlight-example}{%
\subsection{GGHighlight Example}\label{gghighlight-example}}

Plotting with gghighlight is pretty awesome allowing you to filter on
any variable. It seems that gghighlight overwrites any `colour' variable
you put in the main aes. To get round this and have labels, save as a
plot and add geom\_label\_repel separately.

\begin{Shaded}
\begin{Highlighting}[]
\KeywordTok{library}\NormalTok{(gghighlight)}
\KeywordTok{library}\NormalTok{(ggrepel)}

\NormalTok{mydata=gapminder}


\NormalTok{plot =}\StringTok{ }\NormalTok{mydata }\OperatorTok\StringTok{ }
\StringTok{  }\KeywordTok{filter}\NormalTok{(year }\OperatorTok{==}\StringTok{ "2002"}\NormalTok{) }\OperatorTok\StringTok{ }
\StringTok{  }\KeywordTok{ggplot}\NormalTok{(}\KeywordTok{aes}\NormalTok{(}\DataTypeTok{x =}\NormalTok{ gdpPercap, }\DataTypeTok{y =}\NormalTok{ lifeExp, }\DataTypeTok{colour=}\NormalTok{continent)) }\OperatorTok{+}
\StringTok{  }\KeywordTok{geom_point}\NormalTok{()}\OperatorTok{+}
\StringTok{  }\KeywordTok{gghighlight}\NormalTok{(lifeExp }\OperatorTok{>}\StringTok{ }\DecValTok{75} \OperatorTok{&}\StringTok{ }\NormalTok{gdpPercap }\OperatorTok{<}\StringTok{ }\KeywordTok{mean}\NormalTok{(gdpPercap), }\DataTypeTok{label_key =}\NormalTok{ country, }\DataTypeTok{use_direct_label =} \OtherTok{FALSE}\NormalTok{)}\OperatorTok{+}\StringTok{ }
\StringTok{  }\KeywordTok{theme_classic}\NormalTok{()}\OperatorTok{+}\StringTok{ }
\StringTok{  }\KeywordTok{labs}\NormalTok{(}\DataTypeTok{title=} \StringTok{"gghighlight: Filter for countries with Life Expectancy >75 and GDP < mean"}\NormalTok{ )  }

\NormalTok{plot }\OperatorTok{+}\StringTok{ }\KeywordTok{geom_label_repel}\NormalTok{(}\KeywordTok{aes}\NormalTok{(}\DataTypeTok{label=}\NormalTok{ country), }\DataTypeTok{show.legend =} \OtherTok{FALSE}\NormalTok{) }\CommentTok{#only needed if you use  use_direct_label = FALSE. This allows you to have a colour legend as well. }
\end{Highlighting}
\end{Shaded}

\includegraphics{cookbook_files/figure-latex/gghighlight-1.pdf}

\bibliography{book.bib,packages.bib}


\end{document}
